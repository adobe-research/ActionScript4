% Copyright (c) 2014 Adobe Systems Incorporated. All rights reserved.

% Licensed under the Apache License, Version 2.0 (the "License");
% you may not use this file except in compliance with the License.
% You may obtain a copy of the License at

% http://www.apache.org/licenses/LICENSE-2.0

% Unless required by applicable law or agreed to in writing, software
% distributed under the License is distributed on an "AS IS" BASIS,
% WITHOUT WARRANTIES OR CONDITIONS OF ANY KIND, either express or implied.
% See the License for the specific language governing permissions and
% limitations under the License.

\section{New Features}
\label{new}
The following first batch of new features for AS4 provides essentials for:
\begin{itemize}
  \item laying a sound foundation for functional programming in an
  object-oriented setting,
  \item enabling high performance computing.
\end{itemize}

\subsection{The \code{let} Declaration}
In preparation of greater use of concurrency and larger scale
software engineering in ActionScript, and to help program correctness in
general, we strive to make the declaration of constant values easy and to make
immutability the default wherever an explicit indication regarding mutability
is not given.

In AS3, variables are customarily declared with the keyword \code{var} and constants
with \code{const}. However, constants are not really guaranteed to be constant,
i.e. it is often possible to assign a value to them several times and they
sometimes can be read before written to.

In AS4, we replace the \code{const} keyword with \code{let} and we reserve
\code{const} for interesting future uses such as making whole data structures
immutable. We hope that the comparative brevity of \code{let} will encourage
much more frequent usage than \code{var} and we back up the usefulness of
\code{let} with these strong guarantees for every variable declared with it:
\begin{itemize}
  \item Within each active scope and extent, it has exactly one immutable value
  that can ever be observed.
  \item It cannot be read before it is initialized.
  \item Every control flow path leads to an initialization or to a
  state in which it is guaranteed that its value will never be read. The latter
  can for example occur when an exception is thrown between its declaration and
  its initialization.
  \item If there are multiple initializations, then the values they assign can
  differ, but it must be provable at compile time that exactly one will execute
  in every possible control flow that remains in scope.
  \item The compiler ensures all of the above by throwing errors wherever any of
  it does not hold.
\end{itemize}
In short: such constants are truly constant. No value mutation can ever be
programmatically observed. This holds for local variables and for
class instance fields. Even though function parameters are not preceded by
either \code{let} or \code{var}, they are always treated as if they were
declared with \code{let}, i.e. they are truly constant as well.

Both our commitment to promoting immutability and the choice of the keyword
\code{let} instead of \code{const} are deeply rooted in the design philosophy
that AS4 is a multi-style language, which is both object-oriented and
functional. We are aware that \code{const} would be a more familiar keyword for
most of our existing users and that its name immediately indicates immutability.
However, the relevance of these arguments fades quickly with the number of
lines of AS4 code written or read. Looking further ahead, they are outweighed by
the brevity of \code{let}, which matches \code{var} and provides even
indentation, and by its stronger association with functional programming.
This is consistent with our intention to make AS4 surpass today's mainstream
object-oriented languages (JavaScript, Java, C\#, Objective C, ActionScript
3) in this important regard: functional programming is no longer a mere appendix to
object-oriented programming, it is going to be the primary style used for
algorithm encoding, because it is usually superior in performance and
correctness. This holds especially in concurrent settings.

Nevertheless, for well-known reasons, object-oriented code is still
expected to dominate much of programming, as it facilitates programming in the
large. So in the spirit of greater harmony with functional programming,
AS4 provides improved support for both static and instance field immutability.
The above guarantees apply to these as well. Whereas this is straight forward
for static fields, a novel approach is needed for instance fields to prevent
situations like the following:

\begin{verbatim}
class A {
    function f() {
    }

    function A() {
        // compile error in AS4:
        this.f(); // reads x before initialization!
    }
}
class B extends A {
    let x:int;

    override function f() {
        print(this.x);
    }

    function B(x) {
        super(); // calls f()
        this.x = x;
    }
}
\end{verbatim}
Here, the presumable ``constant'' \code{x} is read before the program assigns a
value to it. Other program languages typically offer solutions along these
lines:
\begin{itemize}
  \item The variable \code{x} is already pre-initialized with a default value
  (zero) before any user code runs. This means that two distinct values are observable.
  Hardly a constant! Although static or dynamic optimization can often determine
  that most code only sees the variable after its final assignment, considerable user
  code complications remain.
  \item The above program is rejected by the compiler as it recognizes that
        \code{x} is not a true constant. (This is the case in AS4, but
        see below for why this is problematic and how we solve the problem).
  \item An enhanced (flow-sensitive/context-sensitive) type system lets \code{x}
    be addressable only downstream from its
  initialization statement. In such a system, the above program would still not
  type check.
\end{itemize}
In the latter two cases, the user would be forced to write a factory class. Much
boilerplate! The next section presents our solution to this
important problem.


\subsection{2-Phase Constructors and Truly Constant Fields}
\label{construct2}
In a constructor, every variable field declared by \code{var} is initialized
with a zero-like default value before any user code in the constructor is
executed. This holds even when an explicit initializer is given, since it could
indirectly read the same variable that is about to be initialized.
Constant fields declared by \code{let} can also have an initializer assignment,
but the compiler ensures that they cannot be read during the execution of this
statement.

Constant initialization can be deferred to the constructor body. Still, each
such deferred constant must be initialized exactly once in every possible code
path of the constructor. Example:

\begin{minipage}{\linewidth}
\begin{verbatim}
class A {
    var v :int;
    let dateStamp :Date = Date.current();
    let x :X;

    function A(x :X) {
        this.x = x;
    }
}
\end{verbatim}
\end{minipage}

The variable field $v$ is initialized to zero. Next, the constant field
$dateStamp$ is initialized to the current data. Finally, the constant field $x$
must be initialized in the constructor or the compiler will post an
error.

In order to support the common programming practice of publishing of \code{this}
inside the constructor, constructors are split into two phases. In the first,
default phase, \code{this} can only be used to {\b write} properties of the
object being constructed. Any read operations involving \code{this} are
statically rejected, in particular, \code{this} cannot be passed as an argument,
used to call methods (other than the constructor of the superclass), or aliased
with other variables. In the second, optional phase, the use of \code{this} is
unconstrained.  The second phase is expressed in the \code{defer} block.
Example:

\begin{verbatim}
class B extends A {
    let y :Y;

    function B(x :X, y :Y, k: Key, d :Map) {
        super(x);
        this.y = computeSomething(x, y);
        defer {
            d.add(key, this);
        }
    }
}
\end{verbatim}

\begin{verbatim}
class C extends B {
    let z :Z;

    function C(x :X, y :Y, k: Key, d :Map, z :Z) {
        super(x, y, k, d);
        this.z = z;
        print(d);
        defer {
            print(this);
        }
    }
}
\end{verbatim}

It is of course possible to extend a class with a constructor that has a \code{defer}
block with a class whose constructor does not have one:
\begin{verbatim}
class D extends C {
    function D(x :X, y :Y, k: Key, d :Map, z :Z) {
        super(x, y, k, d, z);
    }
}
\end{verbatim}

The \code{defer} keyword creates a closure that captures the current environment
and schedules this closure to run after all initialization has completed. The
\code{defer} blocks in an inheritance chain are run in inheritance order.

This new language features accomplishes:
\begin{itemize}
  \item No part of \code{this} can ever escape from the initialization part of
  the constructor.
  \item It is thus impossible to observe the value of an uninitialized field.
  \item In the \code{defer} block one can immediately use the constructed object
  without first leaving the constructor context.
  \item For many use cases it is therefore not necessary to write a factory
  method or class.
\end{itemize}

\subsubsection{Constructor Implementation}
It is instructive to inspect in more detail how constructors with \code{defer}
blocks translate to conventional language features. The general form of the
translation scheme, shown below, relies on closures.
However, it is important to note that such closures can always be inlined away,
and an AOT compiler can inline them away at compile time (as illustrated later).

For illustration purposes we pretend here that AS4 already has inline anonymous
closures. Please be aware that this feature will be introduced in a later
version and its syntax may change by then. In reality the compiler creates code
for this behavior directly, without the intermediate step of source code as
shown here.

%Cannot use verbatim mode, because it suppresses the apostrophies
\begin{lstlisting}
static function A.<init>(this :A, x :X) :()=>void {
    // A's init code:
    this.v = 0;
    this.dateStamp :Date = Date.current();
    this.x = x;
    return function():void {
        // A's deferred code
    };
}

static function A.<prep>(this :A, x :X) {
    k = A.<init>(this, x);
    k();
}

static function B.<init>(this :B, x :X, y :Y, k: Key, d :Map) :()=>void {
    k = A.<init>(this, x); // call super
    this.y = computeSomething(x, y); // B's init code
    return function():void {
        k();
        d.add(key, this); // B's deferred code
    };
}

static function B.<prep>(this :B, x :X, y :Y) :()=>void {
    k = B.<init>(this, x, y);
    k();
}

static function C.<init>(this :C, x :X, y :Y, k: Key, d :Map, z :Z)
                        :()=>void {
    k = B.<init>(this, x, y, k, d); // call super

    // C's init code:
    this.z = z;
    print(d);

    return function():void {
        k();
        print(this); // C's deferred code
    };
}

static function C.<prep>(this :C, x :X, y :Y, k: Key, d :Map, z :Z) {
    k = C.<init>(this, x, y, k, d, z);
    k();
}
\end{lstlisting}

And an instantiation statements for each of these classes translate to:

\begin{verbatim}
a = Heap.allocate(VM.instanceSize(A));
A.<prep>(a, x);

b = Heap.allocate(VM.instanceSize(B));
B.<prep>(b, x, y, k, d);

c = Heap.allocate(VM.instanceSize(C));
C.<prep>(c, x, y, k, d, z);
\end{verbatim}

The above implements that in case of \code{c}, these phases execute in this
listed order:
\begin{enumerate}
  \item A's init code
  \item B's init code
  \item C's init code
  \item A's deferred code
  \item B's deferred code
  \item C's deferred code
\end{enumerate}

Both the AOT compiler and the JIT can optimize the above code with directed,
aggressive inlining so that compared to constructors without \code{defer} blocks
no extra call overhead actually occurs at runtime. To arrive at such deep
inlining, the optimizer needs to be capable of relatively simple escape analysis
and of reverting closure abstraction.

In particular, the AOT compiler does not need to emit intermediate closures at
all. For example, the constructor for class \code{C} above can be
inlined to the following:

\begin{minipage}{\linewidth}
\begin{verbatim}
static function C(this :C, x :X, y :Y, k: Key, d :Map, z :Z) :void {
  this.x = x;
  this.y = computeSomething(x, y);
  this.z = z;
  print(d);
  d.add(key, this);
  print(this);
}
\end{verbatim}
\end{minipage}

\subsubsection{Exceptions in 2-Phase Constructors}
What if an exception occurs in a constructor? If it is thrown in the init block
and it is not caught, then it escapes the entire constructor, leaving the
program with no reference to the object being constructed whatsoever. However,
all other side-effects on the rest of the program that already occurred by means
of the constructor's code remain in effect.

If the exception is caught, then its handling code is part of the init block.
The compiler must ensure that also in this case all the above rules for
\code{let}-declared fields apply. Each of them must be assigned exactly once
throughout the entire actual code path and cannot be read. Also, \code{this}
must not escape from exception handlers or \code{finally} blocks.

If an exception occurs in a \code{defer} block and it is not caught in the same
block, then it is propagated all the way out of the constructor. This means
that no \code{defer} block can catch another one's escaping exception. In such
an event, some user-desired invariants may get broken, but at least system
safety remains intact, even if the new object has been leaked, since it is
already fully initialized.

\subsection{Static Verification of Constructors and Constants}

This section summarizes static verification around
\code{super} and \code{let}, to support the above new features.

\subsubsection{\code{super}}

If a \code{super} call does not appear in constructor, a default
\code{super} call (with no arguments)
is inserted at the beginning of the constructor.

Otherwise a \code{super} call
must appear in the beginning of the constructor, and there must not be any other \code{super}
call in the constructor.

\subsubsection{\code{let}}

The compiler needs to prove that a {\tt let} is not read before it is
written, and that it is written exactly once.

At runtime, it should be sufficient to disallow writes to a
{\tt let} through
 dynamic code, and have no restriction on its reads.

\paragraph{Local {\tt let}s}

The compiler ensures that there is one and only one write to a {\tt
  let} as follows:
\begin{enumerate}
\item The \nonterminal{FunctionBody} of a function defined in the same
  scope as the {\tt let}, or any other scope enclosed by that scope, must not
  write to the {\tt let}.
\item In the control-flow graph of the scope, there must not be any
  path from the declaration of the {\tt let} to the end of the scope
  on which there is zero or multiple \nonterminal{Assignment}s to the {\tt let}.
\end{enumerate}

The compiler ensures that there is no read to a {\tt
  let} on any path in the control-flow graph of the scope between the
declaration of a {\tt let} and the one and only one write to the
{\tt let}, as follows:
\begin{enumerate}
\item No function defined in the same scope as the
{\tt let} that reads the {\tt let} or recursively calls such a
function, is read (or called) on the path.
\item There is no read of the {\tt let} on the path.
\end{enumerate}

\paragraph{Static {\tt let}s}

The compiler ensures that there is one and only one write to a {\tt
  let} as follows:
\begin{enumerate}
\item The \nonterminal{FunctionBody} of a function defined in the same
  scope as the {\tt let}, or any other scope enclosed by that scope, must not
  write to the {\tt let}. Furthermore, the {\tt let} must not
  written through a static member reference outside the scope.
 (Note that writes may still be
possible in dynamically typed code: e.g.,
\verb#{dyn:* = A; A.x = ...}#.)
\item In the control-flow graph of the scope, there must not be any
  path from the declaration of the {\tt let} to the end of the scope
  on which there is zero or multiple \nonterminal{Assignment}s to the {\tt let}.
\end{enumerate}

The compiler ensures that there is no read to a {\tt
  let} on any path in the control-flow graph of the scope between the
declaration of a {\tt let} and the one and only one write to the {\tt let}, as
follows:
\begin{enumerate}
\item There is no read, write, or call of anything defined outside the
  class on
  the path, including possibly via dynamically typed code. (Everything
  defined outside the class is assumed to read the {\tt
    let}. We can narrow this assumption in some cases.)
\item No function defined in the same scope as the
{\tt let} that reads the {\tt let} or recursively calls such a
function, is read (or called) on the path.
\item There is no read of the {\tt let} on the path.
\end{enumerate}

\paragraph{Instance {\tt let}s}

The compiler ensures that there is one and only one write to a {\tt
  let} as follows:
\begin{enumerate}
\item The \nonterminal{FunctionBody} of a non-constructor function defined in the same
  scope as the {\tt let}, or any other scope enclosed by that scope, must not
  write to the {\tt let}. Furthermore, a
  \nonterminal{DeferStatement} in a constructor must not write to the
  {\tt let}. Finally, the {\tt let} must not
  written through an instance member reference outside the scope.  (Note that writes may still be
possible in dynamically typed code: e.g.,
\verb#{dyn:* = this; dyn.x = ...}#.)
\item In the control-flow graph of the constructor, there must not be any
  path between the super statement and the defer statement
  on which there is zero or multiple \nonterminal{Assignment}s to the {\tt let}.
\end{enumerate}

The compiler ensures that there is no read to a {\tt
  let} on any path in the control-flow graph of the constructor between the
super statement and the one and only one write to the
{\tt let}, as follows:
\begin{enumerate}
\item There is no read of {\tt this} (other than an access of an
  instance field) and there is no read (or call) of any instance function on
  the path. (Every instance function is assumed to read the {\tt
    let}. We can narrow this assumption in some cases.)
\item No function defined in the constructor that reads the {\tt let} or recursively calls such a
function, is read (or called) on the path.
\item There is no read of the {\tt let} on the path.
\end{enumerate}



\subsection{Fixed Length Arrays}
\label{fixed-length-arrays}
In AS3, the class \code{Array} stands for a versatile yet poorly performing
general purpose collection type with indexing. Often, AS3 performance can
receive a boost by replacing code using \code{Array} with code using
\code{Vector} instead.
And we intend to hold on to \code{Vector} (renaming it to \code{ArrayList}
notwithstanding).
However, in many cases, there is no need for even the few remaining features of
\code{Vector} either.
Often, the length of an array/vector is known to be fixed. Then further code
optimizations can ensue and we can obtain even better performance. This will
hold in particular in the presence of concurrency.

Another reason to introduce fixed length arrays is to increase program
correctness, by encoding a relevant invariant in the type system.

Arrays are not type-compatible with array lists (vectors). Neither type is a
subtype of the other. However, we provide conversion routines between the two
types as described in section~\ref{vector}.

\subsubsection{Declaration and Initialization}
This example declares and initializes an array of \code{int} values of length
$4$.
\begin{verbatim}
    let a :[]int = new [4]int;
\end{verbatim}
The length must not be omitted. Every array must have a definitive length.

All array elements are automatically initialized to the default value of the
element type. In this example, this is zero. For object types, it is
\code{null}. However, an explicit initializer can be given:
\begin{verbatim}
    let a = new []int{computeHeight(), 2, computePrice(weight, count), 4};

    let streetNames = new []String{"Anza", "Balboa", "Chavez"};
\end{verbatim}
The element type must be explicit as shown, and the element count must be
omitted.

In a future release, the type declaration \code{[]int} will be interpreted as
syntactic sugar for \code{Array<int>}. Either notation will then be allowed
interchangeably. However, realizing this immediately would mean precedent
for generalized generics and we would be forced to introduce related features
prematurely.

\subsubsection{Indexing and Length}
\label{arrayIndexLength}
Array elements can be accessed with familiar indexing syntax:
\begin{verbatim}
    value = a[index];
    a[index] = value;
\end{verbatim}
These are the permitted index types: \code{int} and \code{uint}. Each
of them causes the access in question to be translated internally in a slightly different way.
This is disambiguated by the compiler which can regard this as operator
overloading.

The meaning of an array access remains the same across all index types as long
as it succeeds. The reported or inserted value will be the same and the location
addressed in the array will be the same irrespectively. However, bounds checking
mechanics may differ: in case of an unsigned index, checking against indices less than zero is
naturally omitted.

In principle, arrays can have upto $2^{32}-1$ elements. However, runtime memory
management may reduce this at will by rejecting allocations that exceed present
capacity or address ranges. Arrays are thought of as having this method
returning instance length:
\begin{verbatim}
public native final function get length() :int;
\end{verbatim}
We may later introduce an additional array type that has instances of greater
lengths in the 64-bit address range.

\subsubsection{Arrays of Arrays}
Arrays of arrays can be declared by appending additional \code{[]} bracket
pairs. In the following example only the first dimension gets created
and initialized. No elements that represent the additional dimensions are
created at this point. They have to be assigned individually in subsequent
steps.
\begin{verbatim}
    let a2 :[][]String = new [4][]String;
    a2[0] = new [5]String;
    a2[0][3] = "Hello";

    let t :String = a2[1][1]; // null pointer exception
    let s :String = a2[1]; // type error
    let b :[]String = a2[1]; // OK
\end{verbatim}
This change in the construction of the array creates all the substructures as
well:
\begin{verbatim}
    // 4 arrays of arrays with 5 strings each:
    let a2 :[][]String = new [4][5]String;

    a2[0][3] = "Hello";
    let t :String = a2[1][1]; // succeeds
\end{verbatim}
Additional dimensions can be added. This is only limited by resource capacity
and the developers appetite for complexity.
\begin{verbatim}
    let a3 :[][][]String = new [4][5][8]String; // OK
    let a3 :[][][]String = new [][5][8]String; // compile error
    let a3 :[][][]String = new [4][][8]String; // compile error
    let a3 :[][][]String = new [5][4][]String; // OK
\end{verbatim}
Note that the brackets are a type prefix. They denote a type of arrays of what
follows them. This is like in the Go language.

Initializer syntax for arrays of arrays is as follows:
\begin{verbatim}
    let a1 = new []int{1, 2, 3, 4, 5, 6};
    let m2 = new [][]int{{1, 2, 3}, {4, 5, 6}};

    let a3 = new [][][]int{{{{1, f()}, {1, 2}, {1, 2}},
                           {{{1, g()}, {1, 2}, {1, 2}}};
\end{verbatim}

A tree data structure such as an array of arrays differs from a
multi-dimensional array, which is one flat, coherent aggregate without any graph
structure. We will introduce this feature in a later release and we have shaped
the above so that it is forward compatible with our plans in this regard. The
prospect of adding multi-dimensional arrays to the language is the reason we did
not simply define comma syntax as sugar for accessing arrays of arrays.

To preserve the direct applicability of JSON syntax in AS4, we have array
literals such as these:
\begin{verbatim}
    let numbers = [1, 2, 3, 4, 5];
    let names = ["David", "John", "Peter"];
\end{verbatim}
They desugar to:
\begin{verbatim}
    let numbers :[]* = new []*{1, 2, 3, 4, 5};
    let names :[]* = new []*{"David", "John", "Peter"};
\end{verbatim}

\subsubsection{Explanation of Syntax}

The syntax for array types has \code{[]} as prefix (following a modern
trend: see the language Go) rather than postfix (which is
the traditional choice).

Consider what happens with postfix syntax in, say, Java. To initialize
a 4-sized array of \code{String}-arrays, and initialize one of those arrays
to a 5-sized \code{String} array, one may write:

\begin{verbatim}
String[][] a = new String[4][]; a[0] = new String[5];
\end{verbatim}

But this scheme cannot be generalized to arbitrary types instead of
String. For example, if one replaced String by String[], one should
have been required to write the following, which surprisingly is a syntax error!

\begin{verbatim}
var a:String[][][] = new String[][4][]; a[0] = new String[][5];
\end{verbatim}

So Java compromises by requiring a
more complicated interpretation of array type syntax. In other words,
array types are not considered to be of the form \code{T[]}, where
\code{T} could be any type including an array type, but instead array
types are of the form \code{T[]...[]}, where \code{T} is a non-array type.

Unfortunately, the above scheme is brittle.

As we have larger and larger types in the language (function types and
array types in this version, possibly generics in a future version),
it would make sense to have type macros, which would then be expected
to expand seamlessly (without causing syntax errors as above). And moving forward, we may
actually want sizes in array types, like Go (a performance-oriented
language), which opens up better code optimization
opportunities. Thus, we would like to have \code{String [4][5]} in addition
to \code{String[][]} to describe the type of an array. In such a scenario:

\begin{verbatim}
type T = String[5];
... = new T[4]
\end{verbatim}

would macro-expand to:

\begin{verbatim}
... = new String[5][4];
\end{verbatim}

but actually mean:

\begin{verbatim}
... = new String[4][5];
\end{verbatim}

This order reversal is fairly unintuitive. To solve this problem,
Go has \code{[]} as a prefix rather than suffix type operator. This reads
well enough (think: \code{Array<int>}), lets one write \code{new [4][5] String} to
mean 4-sized array of 5-sized \code{String}-arrays, and lets one have a
simple type language where \code{[] T} indeed means ``array of \code{T}'' for all \code{T}, even array types.

There is another, deeper reason (which is related to the above
reasons). Postfix array notation feels natural for programmers who have learned
to parse \code{String[4][5]} en bloc, rather than parsing it as
(\code{String[4]})\code{[5]}. But this idea breaks down when we add another postfix
operator to the language. In particular, we may want to introduce a type of the form \code{T!} in a later version of AS4, that denotes the type of non-null references of type \code{T}. How would that look if we had postfix array notation? Suppose we start with:

\begin{verbatim}
... = new String[4][];
\end{verbatim}

But we want to actually be more specific, and say that we want a 4-sized array of non-null \code{String}-arrays. We would then have to write something like:

\begin{verbatim}
... = new (String[])![4]
\end{verbatim}

This has nothing to do with the fact that we chose \code{!} to be a postfix
operator. If we try to make it a prefix operator, the interaction actually
becomes worse.

A readability argument may be made against the new syntax, but it
mostly comes from brain-print. Following the reasoning above, it is no surprise that other modern languages are beginning to reverse this convention. It might well be that in some years the prefix array notation will become the norm and people will simply get used to it and argue why it feels "readable".

We are not the first to note this problem. As mentioned above, Go
already does this. Other proposals have appeared in the
literature.
For example, C++ type declaration syntax has been rewritten to be
saner.\footnote{\url{http://dl.acm.org/citation.cfm?id=240964.240981}}
 The authors use \code{[][] int} syntax in their cleanup of C++
 declarations, and their overall goals include readability.
As another example, the C type declaration syntax has been explained and some
fixes suggested.\footnote{\url{http://dl.acm.org/citation.cfm?id=947627}} The author uses "array of \code{int}" syntax, the
English equivalent of \code{[] int}, in his systematic explanation of
C's confusing declaration syntax.
(Careful readers of the second paper will note that the simplified
syntax in the appendix leaves array declarations as they are; the
first paper cites the second and states that the second paper
(chronologically first) didn't go far enough.)

In summary, there is no getting around the fact that postfix arrays
make sense only because we don't realize that we are parsing the
dimensions "in reverse" by habit, but that breaks down when we have
anything else that needs to interleave seamlessly with this parsing
rule. The problem is not with postfix per se, but with the fact that
we additionally desire the textual ordering of dimensions to be the
same as the textual ordering of access operations. This problem is
guaranteed to surface when we try to add more type operators to the
language, as we will invariably do to expose more and more
optimization opportunities to the programmer. (That is a design theme
that we are going to adhere to moving forward.)

\subsection{Function types}
One of the larger holes in AS3's type system is that all functions have the same
type, \code{Function}. For almost all uses besides \code{*}-typing, it can be
closed completely in AS4 using the following syntax.
\begin{verbatim}
    function foo(i :int, s :String) :double {...}
    let f :(int, String) => double = foo;
    let d :double = f(1, "Hello");
\end{verbatim}
The parentheses must always be present, even when the parameter list is empty:
\begin{verbatim}
   function bar() :void {...}
   let g :() => void = bar;
   g();
\end{verbatim}
% Also note that \code{void}, which indicates that the function does not return
% anything, must be mentioned explicitly and cannot be omitted.

\subsubsection{Subtyping of function types}

Recall that if a value of type $T$ flows to a variable of type $S$,
then the compiler checks that $T$ is a subtype of $S$, or $T$ is
implicitly convertible to $S$.

A function type of the form
$\code{(}T_1\code{,}\dots\code{,}T_n\code{) => } T$
is considered a subtype of a function type of the form
$\code{(}S_1\code{,}\dots\code{,}S_n\code{) => } S$ whenever
$S_1$ is a subtype of $T_1$, \dots, $S_n$ is a subtype of $T_n$, and
$T$ is a subtype of $S$. This is consistent with the intuition that a
function that takes less precise types and returns a more precise type
can act like a function that takes more precise types and returns a
less precise type.

As usual, a function type is implicitly convertible to and from
\code{*}. Crucially, though, function types that have \code{*} as
parameter or result types are not implicitly convertible to and from
function types of the same shape that have non-\code{*} types in those
positions: such implicit conversion would require wrapping functions
with boxing/unboxing operations, which is deemed too expensive in time
and space. (We might consider this feature later.)

\subsubsection{Default Parameters}

The type of a function with default parameters is a subtype of the function's
full signature, which has all parameters specified as non-default.

\subsection{Type Inference}

In AS4, the programmer need not specify types at all declaration
sites: missing types can be inferred. This is important because of
several reasons:

\begin{itemize}
\item Dynamically typed code is not obligated to be optimized at all by the
compiler and runtime; indeed, dynamically typed code is expected to be
very slow, so the cost of not writing a type in AS4 is
much more than in AS3.
\item As types become more expressive, they also become more verbose (e.g., function
  types and array types in AS4), and it becomes
  more and more tedious to write down static types to ensure good
  performance.
\item With richer numeric types and tighter typechecking around numeric types in AS4, it is
  sometimes better to let the compiler figure out the best possible
  representation for a location rather than specify it explicitly.
\end{itemize}

Type inference
infers static types where possible, and may fall back to inferring the
dynamic type (emitting warnings).

Type inference is limited to the following cases:

\begin{itemize}
\item Type inference can always infer types of \code{let}s.

\item Type inference can always infer types of local variables.

\item Type inference can always infer return types of functions.

\item The programmer must specify parameter types of functions: type
  inference will fall back to the dynamic type for missing parameter types.

\item The programmer must specify types of any non-\code{let} instance variables and
  static variables: type
  inference will fall back to the dynamic type for missing types of non-\code{let}
  instance variables and static variables.
\end{itemize}

This design encourages developers to transform more instance and static
\code{var} declarations to \code{let} declarations, and to leave it to the
compiler to figure out best possible representations for local variables and returns.

Furthermore, this design allows type inference to be implemented by a
simple, standard data flow analysis: the type of
an expression is computed from the types of its parts, and the type
inferred for a location is the union of the types of all expressions
that may flow to that location.

Note that in separate compilation scenarios, it is recommended
practice to annotate types for all non-private fields and non-private
methods of classes, including types for \code{let}s and returns that could have
been inferred.

As an illustrative example, type inference allows the omission of all types other
than parameter types in the following code, without degrading performance.

\begin{verbatim}
function foo(i :int, s :String) // :double
 {
   let x = 2; // x :int
   var r = x; // r :double, since LUB(int,double) is double
   r *= 0.5;
   return r;
 }

 let f = foo; // f :(int, String) => double
 var d = f(1, "Hello"); // d :double
\end{verbatim}

The LUB relation defined earlier for numeric types is generalized to
arbitrary types, as follows.

For reference types $A$ and $B$, the LUB is the least common ancestor
in the inheritance hierarchy.

For function types, the LUB is a function type that takes the
intersection of the parameter types and the LUB of the
result type.

The intersection of two types $T_1$ and $T_2$ is defined as $T$ if $T$
is one of
$T_1$ and $T_2$, and is a subtype of the other. In particular, if
$T_1$ and $T_2$ are numeric types then their intersection is defined
only when $T_1 = T_2$. If
$T_1$ and $T_2$ are reference types then their intersection is defined
only when $T_1 = T_2$ or one is an ancestor of the other in the
inheritance hierarchy.


% \subsection{Property Maps}
% \label{propertyMaps}
% As explained in section~\ref{dynamicClasses} and~\ref{prototypes}, in AS4 we
% no longer support dynamic properties of instances of class \code{object} nor are
% there so-called dynamic classes. However, we want to preserve the ability to
% textually inline JSON data in AS4 programs and we want to facilitate migration
% from AS3 by maintaining the property access syntax that resembles indexing with
% string arguments. Example:
% \begin{verbatim}
% let x = {name : "Player1", age : 18, strength : 98.5}
% x["money"] = 1000;
% if (x["name"] == "Player1") {
%     x["age"]++;
% }
% \end{verbatim}
% The first statement contains a {\em property map literal}. The second adds a
% property. The third reads and tests one. Finally, there is one that updates a
% property.
%
% All this is very similar to dynamic object properties in AS3, but there is no
% conflation with access to static object properties (which are accessed by
% ``.''). These two realms are now cleanly segregated.
%
% The above is supported by the base class \code{PropertyMap}, which has these
% methods:
% \begin{verbatim}
% public class PropertyMap {
%     public final function get(key :String) :* { ... }
%     public final function put(key :String, value :*) { ... }
%     public final function delete(key :String) { ... }
%     public final function contains(key :String) { ... }
% }
% \end{verbatim}
% The indexing syntax is compiler-supported syntactic sugar for calls
% to the former two methods. Here are the same statements as above in their
% desugared form:
% \begin{verbatim}
% x.put("money", 1000);
% if (x.get("name") == "Player1") {
%     x.put("age", x.get("age") + 1);
% }
% \end{verbatim}
% Property maps literals can be used as metadata as shown in
% section~\ref{metadata}.
%
% Note that the familiar AS3 class \code{Dictionary} remains available in
% AS4, mapping \code{Object} keys to \code{*} values.
